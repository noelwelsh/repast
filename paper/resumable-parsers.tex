% -*- coding: utf-8; -*-
\documentclass[english,submission]{programming}
\usepackage{listings}
\usepackage[backend=biber]{biblatex}
\addbibresource{example.bib}
\begin{document}
  \lstset{language=Scala, basicstyle=\small}
  \title{Resumable Parser Combinators}
  \author{Noel Welsh}
  \authorinfo{is the author of this paper. Contact him at
    \email{noel@noelwelsh.com}.}
  \affiliation[a]{Inner Product LLC}
  \keywords{programming journal, paper formatting, submission preparation} % please provide 1--5 keywords
  \paperdetails{
    %% perspective options are: art, sciencetheoretical, scienceempirical, engineering.
    %% Choose exactly the one that best describes this work. (see 2.1)
    perspective=art,
    %% State one or more areas, separated by a comma. (see 2.2)
    %% Please see list of areas in http://programming-journal.org/cfp/
    %% The list is open-ended, so use other areas if yours is/are not listed.
    area={Parsing},
    %% You may choose the license for your paper (see 3.)
    %% License options include: cc-by (default), cc-by-nc
    license=cc-by,
  }
  % \begin{CCSXML}
  % <ccs2012>
  % <concept>
  % <concept_id>10002944.10011122.10003459</concept_id>
  % <concept_desc>General and reference~Computing standards, RFCs and guidelines</concept_desc>
  % <concept_significance>300</concept_significance>
  % </concept>
  % <concept>
  % <concept_id>10010405.10010476.10010477</concept_id>
  % <concept_desc>Applied computing~Publishing</concept_desc>
  % <concept_significance>300</concept_significance>
  % </concept>
  % </ccs2012>
  % \end{CCSXML}

  % \ccsdesc[300]{General and reference~Computing standards, RFCs and guidelines}
  % \ccsdesc[500]{Applied computing~Publishing}

  \maketitle

  % Please always include the abstract.
  % The abstract MUST be written according to the directives stated in
  % http://programming-journal.org/submission/
  % Failure to adhere to the abstract directives may result in the paper
  % being returned to the authors.
  \begin{abstract}
  \end{abstract}


  \section{Introduction}

   Many languages allow string interpolation: a string literal can contain placeholders that indicate where the value of an expression should be substituted into the string. For example, in Scala we can write

  \begin{lstlisting}
    val name = "Noel"

    val hi = s"Hello $name!"
  \end{lstlisting}

  and \lstinline{hi} will have the value \lstinline{"Hello Noel!"}.

  Fewer languages allow extensible string interpolation. Scala is one of them. In Scala the character in front of the interpolated string, \texttt{s} in the example above, determines how the string is processed. The details, which are not important here, are given in the [documentation for the `StringContext` API][StringContext]. Additionally, the result of a string interpolation does not necessarily have to be a string. Interpolation can evaluate to any type. This means that string interpolation can be used to embed domain specific languages (DSLs) within Scala, with interpolation functioning as the interface between the DSL and the Scala host language. Lisp programmers will recognize string interpolation as a form of quasi-quote and unquote.

  This is fine in theory, but there is a problem: how do we parse our embedded DSL when the parsing may be interrupted at any time with an interpolated value? This would be straightforward if only strings could be supplied as interpolated values. In this setting we could simply render everything to a string and then parse the result. However a major advantage of creating an embedded DSL is that we can pass structured data from the host language into the embedded language, and therefore working only with strings is not sufficient.

  In this paper we present \emph{resumable parser combinators}, an extension of parser combinators that allows parsing to be interrupted, parsed values to be injected, and parsing resumed with additional input. We describe the design and implementation of the library. etc.

Four main components:

\begin{enumerate}
  \item model for resumable parser combinators allowing suspending parsing, injecting parsed values, and resuming parsing;
  \item methods to define semantics for suspending and resuming parser combinators;
  \item implementation technique; and
  \item general principles through which we can view the design and implementation.
\end{enumerate}

  \section{Parser Combinators}

...intro sign posts here...

In the abstract we might think of a parser as a function from some input---let's say a \lstinline{String} for simplicity---to some structured value of a domain specific type \lstinline{A}. In a typical case \lstinline{A} will be the type of an abstract syntax tree (AST). A bit of thought will quickly bring us to refine the result type. A parser can fail and usually we are also interested in the remaining unparsed input. So we might say a parser is a function from \lstinline{String} to \lstinline{Result[A]} where \lstinline{Result} is a type that can represent success or failure, and in the case of success can also hold unparsed input in addition to the AST. Listing~\ref{lst:basic-parser} shows a Scala encoding of this representation.

\begin{lstlisting}[frame=lines, caption={A basic \lstinline{Parser} type}, float=*, label=lst:basic-parser]
  type Parser[A] = String => Result[A]

  enum Result[A]:
    case Success(result: A, remaining: String)
    case Failure(reason: String)
\end{lstlisting}

From this basic representation it's natural (at least if you're a functional programmer) to think about composition of parsers. Parsers compose in many ways. For example, sequential composition occurs when one parser consumes the remaining input of another and alternation allows us to try a number of parsers in order until one succeeds. This is the key insight behind \emph{parser combinators} \textbf{CITE CITE CITE}, a well established technique for constructing parsers from simple atomic operations, such as matching a string, and composition operators, such as the above mentioned sequencing and alternation. Parser combinators are simple to implement and use, and many languages have one or more well supported parser combinator libraries.

Cats-parse API examples.

Explicit control over backtracking.

Cats-parse backtracking example.


\subsection{Parsers in Repast}

A standard parser combinator library is at the core of \emph{Repast}. It mostly follows the \emph{Cats Parse} API, so the above examples give a good feel for its use. In particular, it follows \emph{Cats Parse} with a four-state result type and explicit control over backtracking. This type is called \lstinline{Parser}.

\lstinline{Parser} differs from the exposition above in one crucial way: parsers are represented as data structures not functions. This allows us to provide different interpretations for the parsing depending on how we handle interruptions. This change of representation is governed by a simple transformation that we discuss in Section\~\textbf{TODO}. Here we describe the API for controlling

The \lstinline{Parser} type implements a basic parser combinator library. A simple example is given in \ref{lst:digits} to give the flavour of the API.

\begin{lstlisting}[frame=lines,float,caption={A simple parser to recognise a subset of floating point numbers}]
val digits = Parser.charsWhile(_.isDigit)

val float = digits ~ Parser.char('.') ~ digits
\end{lstlisting}



  \section{Design Constraints}

  \section{Design}

  The overall design is summarized by Listing~\ref{lst:parser}.


  \begin{lstlisting}[frame=lines, caption={The \texttt{Parser} type}, float=*, label=lst:parser]
trait Parser[A]:
  def parse(input: String): Parser.Result[A]

  def resume(using
    semigroup: Semigroup[A],
    ev: String =:= A
  ): Suspendable[A, A]

  def resumeWith(f: String => A)(using
      semigroup: Semigroup[A]
  ): Suspendable[A, A]

  def commit[S]: Suspendable[S, A]

object Parser:
  enum Result[A]:
    case Epsilon(input: String, start: Int)
    case Committed(input: String, start: Int, offset: Int)
    case Continue(result: A, input: String, start: Int)
    case Success(result: A, input: String, start: Int, offset: Int)
  \end{lstlisting}
  \printbibliography
\end{document}
